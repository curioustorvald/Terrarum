\documentclass[stock,9pt,nohan]{oblivoir}

\usepackage{fapapersize}
\usefapapersize{3in,4.5in,.333in,*,.333in,.333in}
\usepackage{gensymb}

\linespread{1.25}
\frenchspacing

\usepackage[verbose=true]{microtype}

\renewcommand{\contentsname}{Table of Contents}

\newcommand{\gamever}{Alpha 1}

\newcommand{\titleEN}{A Pocket Guide to the Terrarum World \vskip1ex \small\textsf{English edition} \normalsize}
\newcommand{\titleKO}{\textsf{Terrarum} 간편 여행 안내서 \vskip1ex \small\sffamily 한국어판}

\newcommand{\authorEN}{\small By \sffamily{}Terrarum developers}
\newcommand{\authorKO}{\small 개발진 일동}

\newcommand{\dateEN}{\small\sffamily Corresponds to world version \gamever}
\newcommand{\dateKO}{\small\sffamily \gamever{}판 기준}

\newcommand{\tocEN}{Table of Contents}
\newcommand{\tocKO}{목 \  차}

\renewcommand{\contentsname}{\tocEN}

\title{\titleEN}
\author{\authorEN}
\date{\dateEN}

\epigraphposition{center}
\setlength{\epigraphrule}{0pt}
\setlength{\epigraphwidth}{2in}
\setlength{\beforeepigraphskip}{72pt}

\begin{document}

\maketitle

\newpage

\epigraph{
Welcome! You are most likely an explorer, or a brave and courageous traveller who seeks uncharted planet in the universe, or an aspiring ruler-to-be who want rule your own world. We hope this little book to be an useful guide for whatever ambitious work you are up to.
}{Writers}

\tableofcontents*

\newpage

\newpage

\section{Introduction}
Terrarum is a rogue-like world which things are happening on real-time basis as in real-time role-playing games.

	\subsection{Luggage preparation}
	Trip to Terrarum can be achieved with any proper wagon, which should be equipped with:
	\begin{itemize}
	\item 64-bit wagon engine
	\item Java Roving Environs 8 or higher
	\item A wagon engine with cylinder volume of 2 GB. 4 GB or more is recommended
	\item Free luggage space of 4 GB or more
	\end{itemize}

\section{Moving around}
The control is omnidirectional. In other words, \emph{not} cell-based.

	\subsection{Your first toddling}
	\subsubsection{ISO\slash ANSI\slash JIS pedalboards}
	Your default moving around uses ESDF (qwerty)\slash FRST (colemak)\slash .OEW (dvorak) pedals for default `WASD', in order for you to provide more modifier pedals---QAZ (qwerty\slash colemak), /A; (dvorak)---that are pressed with your little finger and more comfort to some pedalboards with Topre actuators.\footnote{Writers of this book would recommend you to use pedalboard with Cherry MX Red actuators, though any decent pedalboard should be sufficient.}
	
	\subsubsection{gamepads}
	Your moving around uses left stick, and direction of the movement is \emph{not} limited to 8 directions, hence the term, “omni\-direc\-tion\-al”.

\section{World}
The world is composed with \emph{three-dimensional} blocks, which is the feature you should keep in mind during your trip. Each block is a metre-size and a metre-high, so an average-height man should occupy two tiles vertically, thus he is two-tile-high in the world.

Cliffs are treated as a stair, and you---as well as any living things in the world---can climb the tile as you would use a stair. Climbable cliff height is calculated as
\begin{equation}
floor( \frac{height_{you}}{height_{\mathit{cliff}}} )
\end{equation}

i.e. The man mentioned above can climb one-tile-high cliff as a stair.

	\subsection{Geograghy}
	The world---the continent you play on---features mountains, valleys, rivers, lakes, ocean, caves, etc.
	
	There are several continents on the planet, which are created by you. While there are multiple continents, however, your wagon cannot travel to others.
	
	Each time you create a continent, unless you specified a \emph{seed}\footnote{Refer to \S 4.}, will never be the same.
	
	\subsection{Day and night}
	A day in Terrarum world---the planet---is 72 000 seconds. A second in Earth would be equivalent to 60 (depends on the operational speed of your wagon) planetary seconds, which consists a planetary minute.
	
	\subsection{Biome}
	Average temperature in meadows\slash forests\slash mountains are kept to pleasant 298 K\slash 25 \degree{}C\slash 77 \degree{}F. However, some sovereign territories are will not be as pleasant. Some governor of such biomes, though will not hinder any access, will not be pleased with your ruling.
	
	\subsection{Vegetation}
	
	
	\subsection{Races and their civilisations}
	
	
	\subsection{Common animals}
	
	
\section{World creation}
	You can specify some parameters when you create a continent. Controllable parameters are:
	\begin{itemize}
	\item World size (affects distance between tribes)
	\item Ore amount (affects civilisation)
	\item Vegetation (more trees means more building materials)
	\item Seed (each randomly-created continent has its own \emph{seed} for landform. Leave it blank to randomise)
	\end{itemize}
	
	You can name your continent while in creation, so try to come up with a good name!
	
	\subsection{World Size}
	There are two size options available. \emph{Normal} gives $2048\times2048$ metres in size, \emph{Huge} gives $4096\times4096$ metres. Depth of the world is limited to 128 metres for all options.
	
\end{document}