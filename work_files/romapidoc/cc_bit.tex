The Bit API is for manipulating numbers using bitwise binary operations. The ROMBASIC already comes with Lua's bit32 library so make sure to use that for your casual usage.

\section{Functions}

\begin{tabularx}{\textwidth}{l X}
	\textbf{\large Function} & \textbf{\large Notes}
	\\ \\
	\endhead
	bit.blshift(n, bits) & Alias of bit32.lshift(n, bits)
	\\ \\
	bit.brshift(n, bits) & Alias of bit32.arshift(n, bits)
	\\ \\
	bit.blogic\_rshift(n, bits) & Alias of bit32.brshift(n, bits)
	\\ \\
	bit.bxor(m, n) & Alias of bit32.bxor(m, n)
	\\ \\
	bit.bor(m, n) & Alias of bit32.bor(m, n)
	\\ \\
	bit.band(m, n) & Alias of bit32.band(m, n)
	\\ \\
	bit.bnot(n) & Alias of bit32.bnot(n)
\end{tabularx}