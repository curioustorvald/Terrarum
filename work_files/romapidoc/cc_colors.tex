The Colors API allows you to manipulate sets of colors. This is useful in colors on Advanced Computers and Advanced Monitors. British spellings are also supported.

\section{Constants}

When the colours are used in ComputerCraft's Term API, nearest console colours will be used. Below is the table of colours coded with their substitutions.

\begin{tabularx}{\textwidth}{l l l l}
	colors.white & colors.\textcolor{orange}{orange} & colors.\textcolor{magenta}{magenta} & colors.\textcolor{cyan}{lightBlue}
	\\ \\
	colors.\textcolor{yellow}{yellow} & colors.\textcolor{lime}{lime} & colors.\textcolor{tan}{pink} & colors.\textcolor{dimgrey}{gray}
	\\ \\
	colors.\textcolor{brightgrey}{lightGray} & colors.\textcolor{cyan}{cyan} & colors.\textcolor{purple}{purple} & colors.\textcolor{blue}{blue}
	\\ \\
	colors.\textcolor{brown}{brown} & colors.\textcolor{green}{green} & colors.\textcolor{red}{red} & colors.\textcolor{black}{black}
\end{tabularx}

Note that pink is understood as \textcolor{tan}{tan} when it is used, lightBlue and cyan are merged to \textcolor{cyan}{cyan}.

\section{Functions}

All three functions are not supported, as there is no bundled cable thus there is no use of them.