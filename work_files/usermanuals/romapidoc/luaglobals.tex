ROMBASIC adds global functions and constants for operability.

\section{Functions}

\begin{tabularx}{\textwidth}{l l X}
	\textbf{\large Function} & \textbf{\large Return} & \textbf{\large Description}
	\\ \\
	\endhead
	\unemph{\_G.}runScript(\textbf{fun}: str, \textbf{env}: str) & nil & Runs Lua script \textbf{fun} with the environment tag \textbf{env}.
	\\ \\
	\unemph{\_G.}bell(\textbf{pattern}: str) & nil & Strike bell (or beeper) with pattern. See section \emph{Bell Codes} for more information. Aliased to \unemph{\_G.}emitTone.
\end{tabularx}

\section{Constants}

\begin{tabularx}{\textwidth}{l l X}
	\textbf{\large Name} & \textbf{\large Type} & \textbf{\large Description}
	\\ \\
	\endhead
	\unemph{\_G.}\_TERRARUM & non-false & Indicator for multi-environment scripts.
	\\ \\
	\unemph{\_G.}EMDASH & string & EM dash represented by box-drawing character. Code 0xC4
	\\ \\
	\unemph{\_G.}UNCHECKED & string & Unchecked checkbox. Code 0x9C
	\\ \\
	\unemph{\_G.}CHECKED & string & Checked checkbox. Code 0x9D
	\\ \\
	\unemph{\_G.}MONEY & string & Currency symbol used in the world. Code 0x9E
	\\ \\
	\unemph{\_G.}MIDDOT & string & Middle dot used in typography. Code 0xFA (note: 0xF9 is a Dot Product used in Mathematics)
	\\ \\
	\unemph{\_G.}DC1 & string & Ascii control sequence DC1. Used to change foreground colour to black.
	\\ \\
	\unemph{\_G.}DC2 & string & Ascii control sequence DC2. Used to change foreground colour to white.
	\\ \\
	\unemph{\_G.}DC3 & string & Ascii control sequence DC3. Used to change foreground colour to dim grey.
	\\ \\
	\unemph{\_G.}DC4 & string & Ascii control sequence DC4. Used to change foreground colour to bright grey.
	\\ \\
	\unemph{\_G.}DLE & string & Ascii control sequence DLE. Used to change foreground colour to terminal's default error text colour.
	\\ \\
	computer.prompt & string & Default text for prompt input indicator.
	\\ \\
	computer.verbose & bool & Sets whether print debug information to the console.
	\\ \\
	computer.loadedCLayer & table & List of names of compatibility layers has been loaded.
	\\ \\
	computer.bootloader & string & Path to the boot file. Should point to the EFI (/boot/efi).
	\\ \\
	computer.OEM & string & Manufacturer of the computer. If you \emph{are} a manufacturer, you may want to fill in this variable with your own company's name.
	\\ \\
	computer.emitTone(\textbf{len}, \textbf{freq}) & nil & Generates square wave. \textbf{len} is integer, in milliseconds, \textbf{freq} is number, in Hertz.
\end{tabularx}

\section{Bell Codes}

Bell Codes are patterns for driving bell/beeper. Each code is followed by short break of 50 milliseconds.

\begin{tabularx}{\textwidth}{l X l X}
	\textbf{.} (dot) & Short emitTone. 80 ms & \textbf{-} (dash) & Medium emitTone. 200 ms
	\\ \\
	\textbf{=} (equal) & Long emitTone. 500 ms & \textbf{,} (comma) & Short break. 50 ms
	\\ \\
	(space) & Break. 200 ms
\end{tabularx}