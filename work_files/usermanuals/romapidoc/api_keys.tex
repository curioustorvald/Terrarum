The Keys library helps you with Input API to get key code by key's names, or identify a key code.

Notes on compatibility with ComputerCraft: although this library is ComputerCraft-compliant, but Numpads are \emph{not} supported whatsoever. \textit{Come on, it's not like everyone has or likes numpad on their keyboard.}

\section{Functions}

\begin{tabularx}{\textwidth}{l l X}
	\textbf{\large Function} & \textbf{\large Return} & \textbf{\large Description}
	\\ \\
	\endhead
	keys.<key name: String> & int & Returns key code corresponds to the key name.
	\\ \\
	keys.getName(keycode: int) & string & Returns key name corresponds to the keycode.
\end{tabularx}

\section{Accepted Key Names}

\emph{NOTE: following sets are considered as the same keys.}

\begin{itemize}
\item leftAlt --- leftCommand

(leftAlt is often recognised as leftCommand on macOS)

\item leftControl --- capsLock --- backspace

(colemak key layout puts secondary backspace on capsLock, Happy Hacking Keyboard puts Control on the location of Caps Lock)
\end{itemize}

\begin{tasks}[counter-format=\-](5)
	\task (\emph{a} to \emph{z})
	\task (\emph{zero} to \emph{nine})
	\task minus
	\task equals
	\task backspace
	\task tab
	\task leftBracket
	\task rightBracket
	\task enter
	\task leftCtrl
	\task semiColon
	\task apostrophe
	\task grave
	\task leftShift
	\task backslash
	\task comma
	\task period
	\task slash
	\task rightShift
	\task multiply
	\task leftAlt
	\task space
	\task capsLock
	\task scollLock
	\task (\emph{f1} to \emph{f15})
	\task cimcumflex
	\task at
	\task colon
	\task underscore
	\task rightCtrl
	\task rightAlt
	\task pause
	\task home
	\task up
	\task pageUp
	\task left
	\task right
	\task end
	\task down
	\task pageDown
	\task insert
	\task delete
	\task leftCommand
\end{tasks}