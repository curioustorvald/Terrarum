\documentclass[11pt, a5paper]{memoir}

\newcommand*\eiadcrfamily{\fontencoding{OT1}\fontfamily{eiadcc}\selectfont}
\usepackage{inslrmin}
\usepackage[T1]{fontenc}
\usepackage{allrunes}
\usepackage{hyperref}
%\usepackage[fontsize=\mytextsize,baseline=\baselineskip,lines=38]{grid}

% the title page
\title{\textarn{{\huge runaR:fiulkinki \\ +hantbukia+}} \vskip18pt \iminfamily Runes of Arcanum \\ The Practical Guide}
\date{}
\author{}
\hypersetup{
	pdfauthor={ORLY Digital Press / Terrarum},
	pdftitle={},
	unicode=true
}




\begin{document}
\begin{titlingpage}
\maketitle{}
\vfill
\eiadcrfamily O'Really\raisebox{1ex}{\tiny ?} digital press
\end{titlingpage}

\eiadcrfamily

\setcounter{page}{3}

\tableofcontents*

\part*{}

\section{Arcanum Fluxes}
\eiadcrfamily 
Arcanum flux is a quantitive measurement of magical power. Arcanum flux is a basic power unit for spell casting.

Flux must flow through the porst in order to unleash his energie.

\section{Arcanum Ports}
\eiadcrfamily
Arcanum ports are where arcane fluxes are siphoned or poured. Accumulators are where fluxes are stored temporarily. Stored fluxes can be drained to other ports later. Unused fluxes will get dissipated to the environment.

\section{Arcanum Manipulation}
Arcane fluxes can be combined, divided, siphoned, added, multiplied and released. Released fluxes can make all the “magic” happen.

\section{Cost of Power}
Costs of flux vary greatly: the magnitude, proficiency of the caster and the port itself.

\section{Fluctuating Calculation}

\section{Different Ports}

\section{Writing System}

\section{Arcane Language}

\section{Where They Come From}



\end{document}