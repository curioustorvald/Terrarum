\documentclass[stock,9pt,nohan]{oblivoir}

\usepackage{fapapersize}
\usefapapersize{3in,4.5in,.333in,*,.333in,.333in}
\usepackage{gensymb}
\usepackage{allrunes}
\usepackage[T1]{fontenc}

\linespread{1.25}
\frenchspacing

\usepackage[verbose=true]{microtype}

\renewcommand{\contentsname}{\arnfamily efnisifirlit}

\newcommand{\gamever}{\arnfamily alfa:f}

\newcommand{\Terrarumemph}{\arnfamily +iArþin+}

\newcommand{\boktitle}{\arnfamily ferþahantbukin:furiR \\ \Terrarumemph himR \vskip1ex \small nurAna:utkafa \normalsize}

\newcommand{\bokauthor}{\arnfamily \small fra\Terrarumemph hAfuntum}

\newcommand{\bokdate}{\arnfamily\small basa:uiþ:\gamever}

\title{\boktitle}
\author{\bokauthor}
\date{\bokdate}

\epigraphposition{center}
\setlength{\epigraphrule}{0pt}
\setlength{\epigraphwidth}{2in}
\setlength{\beforeepigraphskip}{72pt}

\begin{document}

\maketitle

\newpage

\epigraph{
Uilkumin! You are most likely an explorer, or a brave and courageous traveller who seeks uncharted planet in the universe, or an aspiring ruler-to-be who want rule your own world. We hope this little book to be an useful guide for whatever ambitious work you are up to.
}{Writers}

\tableofcontents*

\newpage

\newpage

\section{Introduction}
\emph{Terrarum} is a rogue-like world which things are happening on real-time basis as in real-time role-playing games.

	\subsection{Luggage preparation}
	Trip to \emph{Terrarum} can be achieved with any proper wagon, which should be equipped with:
	\begin{itemize}
	\item 64-bit wagon engine
	\item \emph{Java Roving Environs 8} or higher
	\item A wagon engine with cylinder size of 2 GB, 4 GB or more is recommended
	\item Free luggage space of 4 GB or more
	\end{itemize}

\section{Moving around}
The control is omnidirectional. In other words, \emph{not} cell-based.

	\subsection{Your first toddling}
	Your default moving around uses ESDF (qwerty)\slash FRST (colemak)\slash .OEW (dvorak) pedals for default `WASD', in order for you to provide more modifier pedals that are pressed with your little finger and more comfort to pedalboards with \emph{Topre} actuators.\footnote{Writers of this book recommend you to use pedalboard with \emph{Cherry MX Red} actuators.}

\section{World}
The world is composed with \emph{three-dimensional} blocks, which is the feature you should keep in mind during your trip. Each block is a metre-size and a metre-high, so an average-height man should occupy two tiles vertically, thus he is two-tile-high in the world.

Cliffs are treated as a stair, and you---as well as any living things in the world---can climb the tile as you would use a stair. Climbable cliff height is calculated as $$ floor( \frac{height_{you}}{height_{cliff}} ) $$

i.e. The man mentioned above can climb one-tile-high cliff as a stair.

	\subsection{Geograghy}
	The world---the continent you play on---features mountains, valleys, rivers, lakes, ocean, caves, etc.
	
	There are several continents on the planet, which are created by you. While there are multiple continents, however, your wagon cannot travel interplanetary.
	
	Each time you create a continent, unless you specified a \emph{seed}\footnote{Refer to Section 4.}, will never be the same.
	
	\subsection{Day and night}
	A day in \emph{Terrarum} world---the planet---is 72 000 seconds. A second in Earth would be equivalent to 60 (depends on the operational speed of your wagon) planetary seconds, which consists a planetary minute.
	
	\subsection{Biome}
	Average temperature in meadows\slash forests\slash mountains are kept to pleasant 298 K\slash 25 \degree{}C\slash 77 \degree{}F. However, you might want to re-think before setting your feet on the snowy area, unless you are prepared well. While the Snow Queen % ---one of the devteam
will not hinder any access to her territory, in the same time she will not be welcoming.
	
	\subsection{Vegetation}
	
	
	\subsection{Races and their civilisations}
	
	
	\subsection{Common animals}
	
	
\section{World creation}
	You can specify some parameters when you create a continent. Controllable parameters are:
	\begin{itemize}
	\item World size (affects distance between tribes)
	\item Ore amount (affects civilisation)
	\item Vegetation (more trees means more building materials)
	\item Seed (each randomly-created continent has its own \emph{seed} for landform. Leave it blank to randomise)
	\end{itemize}
	
	You can name your continent while in creation, so try to come up with a good name!
	
\end{document}